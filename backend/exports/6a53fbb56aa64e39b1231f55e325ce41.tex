
\documentclass{article}
\usepackage[margin=1in]{geometry}
\usepackage{hyperref}
\usepackage{setspace}
\usepackage{titlesec}
\usepackage{lipsum}
\title{Generated Research Paper}
\date{}
\begin{document}
\maketitle
\onehalfspacing

% --- Content ---
Artificial Intelligence in Healthcare

Artificial Intelligence has become a significant force in transforming modern healthcare systems. From diagnosis to treatment planning, AI algorithms help doctors make faster and more accurate decisions. For example, deep learning models can detect pneumonia from chest X-rays and identify tumors from MRI scans within seconds.

AI applications extend to administrative tasks as well. Hospitals now use AI chatbots to schedule appointments, manage patient records, and provide health advice. These tools reduce workload for staff and improve efficiency in patient care.

Despite its benefits, the adoption of AI in healthcare faces several challenges. The most common issues include data privacy, lack of transparency in algorithmic decisions, and the need for high-quality labeled datasets. Overcoming these barriers requires strong collaboration between medical professionals, AI researchers, and policymakers.

In conclusion, AI has immense potential to revolutionize the healthcare industry. With responsible implementation and proper ethical guidelines, it can enhance medical accuracy, reduce costs, and ensure better patient outcomes in the near future.

\end{document}
