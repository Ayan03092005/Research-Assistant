
\documentclass{article}
\usepackage[margin=1in]{geometry}
\usepackage{hyperref}
\usepackage{setspace}
\usepackage{titlesec}
\usepackage{lipsum}
\title{Generated Research Paper}
\date{}
\begin{document}
\maketitle
\onehalfspacing

% --- Content ---
Explainable AI for Disease Diagnosis in Crops
Abstract
This paper presents an Explainable Artificial Intelligence (XAI) approach for detecting and interpreting crop diseases using Convolutional Neural Networks (CNNs). The goal is to enhance farmers’ trust in AI systems by not only identifying diseased leaves but also visualizing the reasoning behind each diagnosis.
1. Introduction
Crop diseases significantly affect agricultural productivity worldwide. Traditional visual inspection is labor-intensive and error-prone. AI-powered image analysis, particularly deep learning, has shown great promise in automating disease diagnosis (Mohanty et al., 2016). However, the “black-box” nature of CNNs limits user trust and understanding. Explainable AI (XAI) bridges this gap by revealing how models make predictions.
2. Methodology
A CNN model (based on ResNet50) was trained on a public plant leaf dataset. Image preprocessing included resizing, normalization, and augmentation. Grad-CAM (Selvaraju et al., 2017) was applied to highlight the regions influencing the prediction, providing visual explanations for the diagnosed disease. Accuracy and interpretability were evaluated jointly to assess both performance and transparency.
3. Results and Discussion
The model achieved 96\% accuracy in classifying major crop diseases. Grad-CAM visualizations clearly indicated leaf regions showing infection patterns, allowing farmers to correlate AI results with real-world symptoms. The addition of explainability improved confidence and acceptance among agricultural experts.
4. Conclusion
The proposed XAI-based CNN system not only detects crop diseases accurately but also provides interpretable insights. Future work includes integrating this system into mobile platforms for real-time field diagnosis and exploring multimodal XAI for integrating environmental data.
References
•	Mohanty, S. P., Hughes, D. P., \& Salathé, M. (2016). Using deep learning for image-based plant disease detection. Frontiers in Plant Science, 7, 1419.
•	Selvaraju, R. R., Cogswell, M., Das, A., et al. (2017). Grad-CAM: Visual explanations from deep networks via gradient-based localization. IEEE ICCV, 618–626.


\end{document}
